% Updated 2024-11-29 15.32
% source: Climate modeling notes 

\chapter{Earth System Models}%slides 16 e 17
until now we've been working with atm models mostly. primitive exercises on ocean models. bc of the role sst was playing, ocean models had to be coupled with atm models. pp didn't stop there.
what about vegetation? i can consider it as external forcing.

CO2 is so well mixed that it doesn't matter where you are, you see true values. Remember $\tau\approx 0.58$ in radiative processes, half of it is caused by CO2.  Can we set up a model that helps us quantify the raising of CO"? This is now an external forcing: it includes the ocean, sst is not external anymore. My odel will still have a portion of internal forcing including ocean coupled var, and the external now is CO2.

big oscillations in the past are ice cores.

CO2 is not the only one: CH4, the graph is in part per bilion.  emission is increasing, not coming from combustion but by cuttle farms, extracting industries with leakagis that are diminushing. also rice production.

climate projections: stone and weaver 2000. given certain social economics hypothesis we have models that predict climate. slide 7. a warmer tem and coolder are equiprob b. redefined what is cold or warm, c. climate change will displace much more warmer than colder over the year. it will not just shift, it will change the character, nonlinear change, that's why we need a model.
this was set up by control run: CO2 present value, increase every year of 1\% then run until eq is reaches double CO2, 4 times CO2 until you reach stab cond. model in a state in which ocean is in eq with atm, otherwise it will drift towards a model with one preferred.