% Updated 2024-11-25 18.26 
% source: Climate modeling notes EXPLORING +SST

\chapter{SST Forcing} %slides 15+cri
How it is able to impact atmospheric circulation?

Let's consider the vorticity equation
\begin{equation}\label{eq. 2.1}
	\frac{\partial \zeta}{\partial t} = - \nabla \cdot (\zeta + f) \mathbf{V} = - (\zeta + f) \nabla \cdot \mathbf{V} - \mathbf{V} \cdot \nabla (\zeta + f) = - (\zeta + f) D - \mathbf{V} \cdot \nabla (\zeta + f)
\end{equation}

Where D is the divergence and it is equal to $$D=u_x + v_y = w_z = \frac{\partial w}{\partial z}$$
and the last term is the advection.

If we linearize \ref{eq. 2.1}, all the non linear terms will be thrown away so that

$$\frac{\partial \zeta}{\partial t} = - f D - v \frac{\partial f}{\partial y} = - f D - \beta v $$
The equation above stresses that is due to these two terms that vorticity evolves.

Assuming steady motion $$\frac{\partial \zeta}{\partial t} = - f \frac{\partial w}{\partial z} - \beta v$$  becomes $$f \frac{\partial w}{\partial z} = \beta v$$



what is the mechanism by which SST is affecting climate. SST has been the main factor in understanfing the long term variability (beyond synoptic timescales of weeks), where the impact of the SST becomes important.
the eq for the temprature of the surface: rate of change of t is the resultig balance of heat emission, in the ocean is more difficult, the adjustment is not easy. Surface t on land doesn't keep memory. Soil moisture maintains the memory: 1 year to let the circulation adjust --> much more difficult than ocean surface temperature. How is the SSt capable of affecting the circulation? --> vorticity balance. What happens when we have forcings --> in the Eq zones. SST linked to convection. Localized heating sources --> asymmetry in E-W, the response to the fact that heating causes air rise. cyclons in the lower and upper level: heating is forcing vorticity from vorticity balances. if you heat smt, air rises and you get vorticies in the upper atm, displaiced from the heating slide 11

Same problem but with a basic state: constant wind with dissipation (compensate for the lack of adjustment). You can find a solution. Barotropic system so let's apply the forces by the mountain: Grose and Hoskins. the consclusion is that mountains can cause Rossby waves. impact on the basic state with added anomaly slide 17.
(rossby waves are planetary waves).
because we want stationary sol: linear dissipation term i vorticity and in temperature (raduatuve cooling). Forcing a system you're adding energy: you need to find a way to get rid of it.

Hoskins and Karoly 1981. Stationary heating and looked at response. GCM was a realistic one. The model was linearized. The nonlinear sol are stationary sol. . first i create a complex math set up, in order to find out the response for the heating i need to semplify and solve the linear problem coming from it. How do i find solutions? once you discretize (grid point or spectral resolution) is just a linear system.  Let's use 2degree resolution, grid point180/2x360/2=18000 points. Manabe modellized 9 levels, hence 9x1800 points, then I'll have 60 000 unknown --> matrix 600 000 bu 600 000. one way is computing the matrix A and solve the system $A\mathbf{x}=f$. Another way. You already have the model, $\mathbf{\dot{x}}=F(\mathbf{x})$ taking all the prognostic variables and make the time der, I could simply solve $\mathbf{\dot{x}}=F(\mathbf{x})+Q$, but i don't have F, I need the linearized version of it. I could use the models. I let the model take one timestep, plus a small perturbation in one grid point. I choose the perturbation x' in a way that is a special vector: 1 0 0 0 0 ... and I taje the der that is just the basic state, this der will be the column of A in the place i put 1 perturbation. I can construct the matrix: it hase the diagonal blocks in each m. you need a number of timesteps equivalent to the degrees of freedom = grid points/spectral coefficients of divergence, vorticity, ... (unknown you have in your disretization).



Which term balances the heating? the heating goes all the way to the vertical velocity or advections. Question is: is it coing to vertical adv or horizontal adv? we don't know. How big is beta and how trong my stratification? the mechanism generating the smaller advection will dominate as it's the easiest to treat. in teh end you come up with a parameter that indicates you what kind of advection you have--> $\gamma$.  at the eq if you heat smt you generate the vertical motion, this heating is due to deep convection, generated by SST  (if SST>28.5° strong convection).
slide 23 balance elsewhere.  vorticity in both layers but opposite signs--> generation of vorticity in two layers. anomalous heating has completely different consequences in north atlentic or euator, just because of vorticity balances.

slides 25 looks like Horance and Wallace paper in SST variability. the sources may be anolamous heating generating anomalous circulation. AUDIO. teleconnections might have dyn causes on a timescales of couple of months.
Assumption was that basic states have simple structure, but they don't.

anomalies are growing on the gradients.